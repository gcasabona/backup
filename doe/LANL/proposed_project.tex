\documentclass[12pt]{article}
\usepackage{amsmath}
\usepackage[margin=1in]{geometry}
\usepackage{xcolor}
%\parindent = 0.0in

\newcommand{\todo}[1]{{\color{red}{\it $\langle$todo$\rangle$} #1}}

\begin{document}

%%%%%%%%%%%%%%%%%%%%%%%%%%%%%%%%%%%%%%%%%%%%%%%%%%%%%%%%%%%%%%%%%%%%%%%%%%%%%%
\subsection*{Provide a brief description of your proposed practicum project:
(3000 character max)}


Neutron star mergers are catastrophic cosmic events that are exciting for
multiple reasons: they generate gamma-ray bursts -- brilliant flashes of 
high-energy electromagnetic radiation; they produce powerful gravitational
waves, detectable from billions of light-years away; and they are most likely
the primary source of heavy r-process nucleosynthesis in nature. Recently, a
few of them were discovered by gravitational wave detectors, LIGO and VIRGO, along with
electromagnetic counterparts and high-energy neutrinos.
Now, more accurate gravitational-wave templates are needed to decipher these
and future detections.

Neutron stars bear a solid crust, which is anticipated to have exceptional
strength. Studies show that the crust can leave an observable imprint on the
gravitational wave signal. In this practicum project, it is proposed to
numerically examine how the crust affects tidal deformability of a neutron star 
in equilibrium. Specifically, a neutron star with a crust will be prepared in
an equilibrium state, and then subjected to a periodic quadrupolar perturbation,
mimicking the gravitational field imposed from a companion in a binary system. The effect of the
perturbation will be measured and employed for calculating corrected tidal
deformability. The latter can then be plugged into post-Newtonian
expressions for gravitational waves of a neutron star binary inspiral.

The proposed study will be conducted using SPaRTA, a novel hybrid OpenMP/MPI,
fully general-relativistic framework. This code is currently being developed
at Los Alamos National Laboratory, and will be used for simulating neutron
star mergers. The latter includes an adaptive curvilinear multi-block grid that
can be fine-tuned to resolve the crust with very high accuracy.

As a first step, we will work on the design of a crust-conforming
curvilinear multi-block grid. We envision that the so-called ``cubed sphere''
configuration is best suited for this problem. We will then work on
load-balancing this configuration and optimizing the grid resolution for best
performance. Finally, we will perform several fully general-relativistic 
medium- and high-resolution runs of realistic configurations of neutron stars 
subjected to periodic quadrupolar perturbations. Our findings will be
summarized in a publication.

%%%%%%%%%%%%%%%%%%%%%%%%%%%%%%%%%%%%%%%%%%%%%%%%%%%%%%%%%%%%%%%%%%%%%%%%%%%%%%
\subsection*{Describe the relationship, if any, of proposed practicum work
with your thesis work: (2400 character max)}

The main similarity between this project and my thesis work is that they both explore
areas of general relativity. My thesis work is based on numerical relativity modeling, purely focusing on gravitational waves, whereas this practicum project will focus on the multi-physics problem of resolving the crust of the neutron-star, electromagnetic radiation of the breaking of the crust, and the exotic fluid-matter that is theorized to exist in the interior of the neutron star. 

%%%%%%%%%%%%%%%%%%%%%%%%%%%%%%%%%%%%%%%%%%%%%%%%%%%%%%%%%%%%%%%%%%%%%%%%%%%%%%
\subsection*{List computational resources required for practicum project, if
any: (2400 character max)}

\todo{}

%%%%%%%%%%%%%%%%%%%%%%%%%%%%%%%%%%%%%%%%%%%%%%%%%%%%%%%%%%%%%%%%%%%%%%%%%%%%%%
\subsection*{Are you planning on utilizing any of the HPC resources at the
lab? (2400 character max)}

\todo{}

\end{document}


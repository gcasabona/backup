\chapter{Introduction}

Type Ia supernovae (SNe Ia) are the thermonuclear explosions of white dwarfs (WDs). These WDs are in binary systems, accreting matter from their companion stars. One of the many unique characteristics of SNe Ia is that they have a consistent peak luminosity, which is why they are used in cosmology as standardizable candles \cite{phillips93}. Their use as standardizable candles helped in the discovery of the accelerated expansion of the universe \cite{Riess98}. SNe Ia are also prominent sources of cosmic rays and the abundance of $^{56}$Fe in the universe.

Although it is known that SNe Ia are the thermonuclear explosions of WDs, their detonation mechanism remains unknown. In our previous work \cite{Fisher}, we proposed a detonation mechanism for carbon in electron-degenerate matter due to turbulence. Using localized 3-dimensional hydrodynamics simulations, we found that carbon can indeed detonate in electron-degenerate turbulent matter in the distributed burning regime. We refer to this new mechanism as a \textit{turbulently-driven detonation mechanism}. For this project, we explore the parameters of how the detonation of helium may lead to the detonation of carbon. Inspiration comes from previous literature that determined what role the detonation of the helium surface of the WD plays in SNe Ia.  

\section{Type Ia Supernovae}

The categorization of SNe begins with hydrogen. Type II SNe are characterized by strong hydrogen absorption lines in their spectrum. SNe II result from the collapse of a massive star, with masses greater than 10 M$_{\odot}$, at the end of their lifetime. Type I SNe have no hydrogen lines in their spectra. Furthermore, SNe Ia spectra have strong silicon absorption lines. Early models predicted that SNe Ia must come from a compact object, which was confirmed by SN 2011fe \cite{Nugent_2011}. Astronomers were able to observe the region where the event took place before and after the explosion, confirming early suspicions. The only two compact objects in the universe that can explode, as far as we know, are WD and neutron stars (NS). NS are known to explode into kilonovae, so strong confidence is put into SN 2011fe originating from a WD.

Progenitors of SNe Ia involve a primary WD in a binary system. In a single-degenerate channel, the secondary is usually a star still on main sequence. Extensive work has gone into exploring single-degenerate models, however, observational evidence does not support this channel since no surviving companion main sequence star has yet been found for a normal SN Ia event.

In the double-degenerate (DD) channel, both the primary and secondary stars are WDs. The primary WD tidally disrupts the secondary WD, creating an accretion disk around the primary. Over time, this disk accretes onto the primary WD, creating a highly turbulent environment. Motivated by the DD channel, our previous work showed that the turbulent cascade caused by the accreting matter can lead to a detonation \cite{Fisher}. Those models included electron-degenerate matter consisting of a 1:1 ratio of carbon and oxygen. For this project, we take this mechanism one step further to explore how turbulence can lead to the detonation of helium, which might then detonate carbon.   

\section{Helium Detonation Models}

WD are known to have a relatively thin helium shell around them resulting from stellar evolution \cite{giammichele}. During the merger of these binary systems, the helium of the secondary WD will accrete first and mix with the helium layer of the primary WD. Motivation for this project comes from recent literature which explores various mechanisms of the detonation of this helium layer and how this leads to the detonation of the carbon core. One of these mechanisms is called the double-detonation mechanism. In one possible scenario, two individual spots in the helium layer of the primary WD detonate, sending shock waves radially inwards towards the carbon core. At the point where the shock waves meet, carbon detonation is intitiated at an off-center location. 

In a more realistic model, the helium accretion from the secondary WD continuously adds to the helium shell of the primary WD, until the primary detonates \cite{Shen}. This mechanism adds more kinetic energy since mass is being added into the system with high velocities. Known as the dynamically driven double-degenerate double-detonation, D$^6$, SNe Ia are now modeled with a much wider range of initial conditions. An outcome of the D$^6$ model is complete detonation of the primary WD, which sends the secondary WD as a hypervelocity runaway. Recent observations from \textit{Gaia} now support the existence of this new model \cite{Gaia}, further increasing the need to explore this detonation mechanism.

\section{Turbulence}

The majority of the universe exists in a fluid state, either gas or plasma. For this reason, the study of fluid dynamics is crucial for understanding phenomenon happening in the universe. In astrophysics, modeling fluids takes the form of Euler's equation,
\begin{equation*}
	\frac{\partial (\rho \textbf{v})}{\partial t} + \nabla \bullet (\rho \textbf{vv}) = -\nabla P - \rho \nabla \Phi,
\end{equation*}
where $\rho$ is the mass density, $\textbf{v}$ is the velocity vector, $P$ is the thermal pressure, $\Phi$ is the gravitational potential, and $G$ is the universal constant of gravitation \cite{euler}. In this form, the  fluid has zero viscosity and includes effects from self-gravity. This project involves modeling electron-degenerate matter at scales much higher than those needed to have viscosity play a role, so the approximation of the Euler fluid will suffice.

In the simplest of cases, fluids are modeled as a laminar flow. This means that the velocity vector lines are all smoothly varying. Realistic cases, however, need to include turbulence, in which the velocity vector lines are no longer smooth. This is caused from chaotic changes in the fluid's velocity and pressure. The onset of turbulence typically begins with some kind of fluid instability. Rayleigh-Taylor and Kelvin-Helmholtz are two of the major instabilities found in astrophysics.

An important quantity that is needed when describing turbulence is the dimensionless Reynold's number, (Re), defined as
\begin{equation*}
	{\rm Re} = \frac{\rho u L}{\mu}.
\end{equation*}
Here, $\rho$ is the density of the fluid, $u$ is the velocity of the fluid, $L$ is the linear dimension, and $\mu$ is the dynamic viscosity of the fluid. Re above 2000 means that the fluid flow becomes turbulent. Astrophysical fluids have low viscosities, resulting in them almost always being turbulent. An important quantity here is the linear dimension, meaning that the Re number is dependent on the length scales in question, a property which will be exploited later. 

One important aspect of turbulent flow is the onset of a turbulent cascade. Kolmogorov's theory of turbulence tells us that the turbulent cascade allows for the cascade of energy. When turbulence is first initiated, its eddies are at the largest length scales, which is determined by the geometry of the system. The time-scale of these eddies is determined by its turnover time, the time it takes for one eddy to make one complete loop. As time evolves, these eddies break up and form smaller ones, which then have smaller time-scales. Once this time-scale is small enough, the viscosity of the fluid then dissipates the kinetic energy of the turbulence into heat. In our previous work, we showed that when the time-scale of the eddies reduces even lower until equaling that of the nuclear burning time-scale, this is the moment of the initiation of detonation \cite{Fisher}. Another important aspect of Kolmogorov's theory is that the turbulence becomes isotropic as the time-scales are reduced, for high Re numbers. This means that regardless of the large scale eddies, which is determined by the geometry of the system, the statistics of initial range turbulence are universal.


\section{Distributed Burning Regime}

In a regime with negligible turbulence, the nuclear burning is laminar. This flame is characterized by a length, $l$, and a speed, $s_l$. The flame maintains a well-defined structure since turbulence has little effect on it. When turbulence begins to influence the structure of the flame, we then consider the dimensionless Karlovitz number, (Ka), defined as
\begin{equation*}
	{\rm Ka} = \sqrt{\frac{v_{\rm RMS}^3}{s_l^3} \frac{l}{L}},
\end{equation*}
where $L$ is the integral scale and $v_{\rm RMS}$ is the RMS velocity at the integral scale. When Ka < 1, turbulence is low enough that the structure of flame remains intact. For Ka > 1, turbulence begins to disrupt the structure of the flame. For Ka $\gg 1$, the flame is completely disrupted by the turbulence. The flame now exists in the distributed regime, where the turbulent mixing dominates electron conduction. In regions of key interest in the double-degenerate channel of SNe Ia, Ka $\simeq 10^4$, deep into the distributed burning regime, another motivator of this project.

\section{Zel'dovich Gradient Mechanism}

To better understand the significance of the distributed burning regime, it is important to note what the leading theory was in detonation mechanisms for SNe Ia. First modeled by Zel'dovich and collaborators \cite{zeldovichetal70}, the Zel'dovich gradient mechanism describes how a laminar flame may lead to a detonation. It begins with the formation of a laminar flame in a hot background, accelerating down a temperature gradient. The formation of the flame initiates a shock front ahead of it. If the temperature gradient is shallow enough, then this initially subsonic laminar flame will accelerate just behind the shock front. The shock front leading the flame front allows for the carbon in that regime to fully burn, causing a detonation. When the temperature is too steep, the shock front will accelerate much quicker than the flame front, leaving it behind. In this case, carbon is not fully burned, causing a failed detonation. Failed detonations with the Zel'dovich gradient mechanism put a lower limit on the temperatures needed to detonate carbon. Our previous work shows with turbulence taken into consideration, critical temperatures for carbon detonation are a factor of 2-3 times lower than previous studies based upon Zel'dovich. 

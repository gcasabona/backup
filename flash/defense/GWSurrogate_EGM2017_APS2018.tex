%  Used for...
%
%  1) APS New England Fall 2018 -- intro to NRHybSur3dq8 and RIFT
%
%
%

%\documentclass[compress]{beamer}
\documentclass[uncompress,aspectratio=43]{beamer}  % APS

\usetheme{Boadilla}
%\usecolortheme{dolphin}

% to wrap text around a figure
\usepackage{wrapfig}

\usepackage{tikz}

% to wrap text around a figure
\usepackage{wrapfig}

\usepackage{setspace}

% to split the top headliner
%\usepackage{beamerthemesplit}

\usepackage{multirow}

\usepackage{dcolumn,epsfig}

% for side captions (doesn't seem to work, neither does minipage. must use column environment)
\usepackage{sidecap}

% to displace animated graphics
\usepackage{animate} %need the animate.sty file 
\usepackage{graphicx} %needed if animating and creating pdf through latex -> dvi -> pdf

\usepackage{algorithm}
\usepackage{algpseudocode} % in place of algorithmic. Appears to conflict with revtex4 unless [H] option passed

%allows comments for blocks of text
\usepackage{verbatim} %with begin/end comment

% fancy version of verbatim
%\usepackage{fancyvrb}

\usepackage{listings}

% for fancy box options
\usepackage{fancybox}

\usepackage{tabls}
\usepackage{subfigure}

\usepackage{mathrsfs}

\newcommand{\red}{\textcolor{red}}
\newcommand{\blue}{\textcolor{blue}}
\newcommand{\green}{\textcolor{green}}
\newcommand{\ljump}{\big[\!\big[}
\newcommand{\rjump}{\big]\!\big]}
\newcommand{\half}{{\textstyle \frac{1}{2}}}
\newcommand{\indspace}{\hspace{6pt}}
\def\vecmu{{\vec{\mu}}}
\def\MMm{ {\mathrm{MM}} } 
\newcommand{\ta}{\ensuremath{\tilde{A}}}
\newcommand{\cg}{\ensuremath{\tilde{\gamma}}}
\newcommand{\complex}{\mathbb{C}}
\newcommand{\real}{\mathbb{R}}
\newcommand{\cP} {{\cal P}}
\newcommand{\cF} {{\cal F}}
\newcommand{\cT} {{\cal T}}
\newcommand{\cC} {{\cal C}}
\newcommand{\cH} {{\cal H}}
\newcommand{\nwidth}[2]{d_{#1}}
\newcommand{\greedy}[2]{\sigma_{#1}}
\newcommand{\triplenorm}[1]{\left| \! \right| \!\! | {#1} | \!\! \left| \! \right|}
\newcommand {\cM} {{\cal M}}
\newcommand {\cA} {{\cal A}}

\newfloat{program}{htbp}{pgm}
\floatname{program}{Program}

%\xdefinecolor{eqncolor}{rgb}{0,0.35,0}
\xdefinecolor{eqncolor}{rgb}{0,0,0}
\xdefinecolor{ec}{rgb}{0,0.35,0}

\definecolor{bg}{rgb}{0.95,0.95,0.95}

% fancy code -- see: http://tex.stackexchange.com/questions/176163/problem-in-adding-a-title-to-tcolorbox
% can pass general options to minted: %\begin{pythoncode}[colback=red!5!white,colframe=red!75!black,title=My nice heading]    
\usepackage{tcolorbox}
\tcbuselibrary{minted,skins,breakable}
\newtcblisting{pythoncode}[1][]{    
  listing engine=minted,    
  breakable,   
  colback=bg,    
  colframe=black!70,    
  listing only,    
  minted style=colorful,    
  minted language=python,    
  minted options={linenos=true,numbersep=3mm,texcl=true},    
  left=5mm,enhanced,    
  overlay={\begin{tcbclipinterior}\fill[black!25] (frame.south west)    
            rectangle ([xshift=5mm]frame.north west);\end{tcbclipinterior}},
            #1,
}    
\newtcblisting{ipython}[1][]{    
  listing engine=minted,    
  breakable,   
  colback=bg,    
  colframe=black!70,    
  listing only,    
  minted style=colorful,    
  minted language=python,    
  minted options={numbersep=3mm,texcl=true},    
  left=5mm,enhanced,    
  overlay={\begin{tcbclipinterior}\fill[black!25] (frame.south west)    
            rectangle ([xshift=5mm]frame.north west);\end{tcbclipinterior}},
            #1,
}   

\setbeamersize{text margin left=.05cm,text margin right=.05cm}

\title[PE using NR Surrogates and RIFT]{Parameter estimation of binary black hole systems using numerical relativity surrogates and a rapid inference framework}

\author[Feroz]{Feroz Hussain Shaik}
\institute[UMass Dartmouth]{Dept. of Physics, UMass Dartmouth}
\date[APS NES \today]{{APS New England Fall 2018 Meeting\\
\today}
% \\[1cm]
%\scriptsize
%Joint work with
%Scott Field (U. Mass Dartmouth), Richard O'Shaughnessy (RIT), 
%Vijay Varma (Caltech), Jacob Lange (RIT)
%\\[.5cm]
%{arXiv: [gr-qc]}
}
\scriptsize

\begin{document}

\frame{\titlepage
%   \vspace{-25pt}
%\begin{minipage}[b]{\linewidth}
 % \vspace{20pt}
%  \raggedright (arXiv:1304.0462)
%\end{minipage

\vspace{-39pt}

  \begin{columns}[T]
\hspace{10pt} 
    \begin{column}{0.55\textwidth}
%{\scriptsize
%\noindent \blue{Collaborators}: Lawrence Kidder, Francois Foucart, Erik Schnetter,
%Saul Teukolsky, Andy Bohn, Nils Deppe, 
%Peter Diener, Francois Hebert, Jonas Lippuner, 
%Jonah Miller, Christian Ott, Mark Scheel, Trevor Vincent}
%\includegraphics[width=.95\textwidth]{/home/balzani57/GlobalBackUp/Research/talks/CommonSlides/sxs_logo_horizontal_4240px.png}

    \end{column}
    
    \begin{column}{0.45\textwidth}
      
     \hspace{50pt} \includegraphics[width=.36\textwidth]{/home/frz/WORK/beamer/UMassDartmouthLogo.jpg}
  
    \end{column}
    
  \end{columns}

}


\AtBeginSection[]
{
  \begin{frame}<beamer>
		\frametitle{\Large{Outline}}
		\tableofcontents[currentsection]
                %\tableofcontents[currentsection,subsectionstyle=hide]
  \end{frame}
}

\begin{frame}
Ongoing MS Thesis Project supervised by 
  \begin{itemize}
    \item Scott Field (UMass Dartmouth)
    \end{itemize}
\medskip
   Joint work with
   \begin{itemize}
   \item Jacob Lange (RIT) (RIFT+ILE)
   \item Richard O'Shaughnessy (RIT) (RIFT+ILE)
   \item Vijay Varma (Caltech) (NR Surrogate Model)
  \end{itemize}

\end{frame}

%\begin{frame}{...and {\em many} NR simulations}

%\begin{columns}
%  \begin{column}{.45\linewidth}
%\begin{figure}
%\includegraphics[width=.95\textwidth]{/home/balzani57/GlobalBackUp/Research/talks/CommonSlides/sxs_logo_stacked_3775px.png}
%\end{figure}
%  \end{column}
%  \begin{column}{.45\linewidth}
%\includegraphics[width=.75\textwidth]{/home/balzani57/GlobalBackUp/Research/talks/CommonSlides/SpECViz.png}
%  \end{column}
%\end{columns}

%\end{frame}

\begin{frame}
\frametitle{Introduction}

\medskip
GW Parameter Estimation
\begin{itemize}
\item Figuring out what are the properties of the binary sources using the data
\item Usually minimum 15 parameters required
\end{itemize}

\pause
\medskip
Parameters include
\begin{itemize}
\item Masses and spins of black holes 
\item Luminosity distance to binary $D_L$
\item Inclination angle $\iota$
\item RA and dec (position in sky)
\item ..and so on
\end{itemize}

\end{frame}

\begin{frame}{Introduction}
  Why do we need parameter estimation?
\begin{itemize}
\item Accurate parameter estimation required for science (eg. population
  studies, tests of GR)
\item Fast algorithms can help follow-up observations

\end{itemize} 
\end{frame}

\begin{frame}{Bayes' Theorem}

Updating and maximizing our knowledge from the data and our prior belief
\begin{center}
  $$
  p_{post}(\lambda, \theta) =
  \frac{\mathcal{L}(\lambda, \theta) p(\theta)p(\lambda)}
  {\int d\lambda d\theta \mathcal{L}(\lambda, \theta) p(\lambda) p(\theta)}
  $$
\end{center}
  
\begin{itemize}
\item Posterior $p_{post}$ - probability of parameter value to be correct given the
  data. This is what we are looking for
\item Likelihood $\mathcal{L}$ - probability of data assuming model is true i.e.
  how well does the model parameter fit the data
\item Prior $p(\theta)p(\lambda)$ - what we think the model parameter PDF looks like
\item Evidence - normalization constant, important for model selection
\end{itemize}

\end{frame}

\begin{frame}{Bayes' Theorem}

  Updating and maximizing our knowledge from the data and our prior belief
\begin{center}
  $$
  p_{post}(\red{\lambda}, \blue{\theta}) =
  \frac{\mathcal{L}(\red{\lambda}, \blue{\theta}) p(\blue{\theta})p(\red{\lambda})}
  {\int d\red{\lambda} d\blue{\theta} \mathcal{L}(\red{\lambda}, \blue{\theta}) p(\red{\lambda}) p(\blue{\theta})}
  $$
\end{center}
  
\begin{itemize}
\item \red{$\lambda$} - Intrinsic parameters, describes dynamics of binary
\item \blue{$\theta$} - Extrinsic parameters, describes space and time orientation
\end{itemize}

\end{frame}
\begin{frame}{Models}
Templates for aligned spin 
  \begin{itemize}
  \item SEOBNRv4 - Effective-one-body model
  \item IMRPhenomD
    \item Numerical Relativity (NR) simulations - solving Einstein's equations on
      a computer. Very expensive to compute, but gives most accurate results
\pause
    \item Surrogate models
\end{itemize}
\end{frame}

\begin{frame}{Surrogate Models}
  \begin{itemize}
    \item Fast and accurate evaluation using reduced order modeling (ROM)
  \item No assumptions of underlying model
    \item Will converge to the model by improving representation
    \item Need to specify model and range of parameters
    \item Non-intrusive
      \item gwsurrogate - Publicly available package
        
\end{itemize}

\end{frame}

  \begin{frame}{NR Surrogate Models}
    NRHybSur3dq8
    \begin{itemize}
\item Aligned spin hybrid model made by stitching together PN for early inspiral, NR simulation
  for the rest
\item Contains higher modes upto lmax $\le$ 5
\item Training region: 1 $\le$ q $\le$ 8, −0.8 $\le$ $\chi_{1z/2z}$ $\le$ 0.8. 
\end{itemize}
\end{frame}

\begin{frame}{NRHybSur3dq8}
  \begin{figure}
    \begin{center}
      \includegraphics[width=.70\linewidth]{ModelErrors.png}
      \caption{Errors in the NRHybSur3dq8 model against NR-PN/EOB hybrid
        waveforms.
        From Varma et al. (2018, in prep)}
    \end{center}
  \end{figure}
\end{frame}

\begin{frame}{NRHybSur3dq8}
    \begin{figure}
    \begin{center}
      \includegraphics[width=.70\linewidth]{SurrogateTrainingSet.png}
      \caption{.The parameter space covered by the 104 NR waveforms (circle markers) used in the construction of the surrogate model. From Varma et al. (2018, in prep)}
    \end{center}
  \end{figure}
\end{frame}

\begin{frame}{Inference algorithms}
  LALInference (LIGO)
  \begin{itemize}
  \item Using variants of optimized MCMC and nested sampling
  \item Well-tested and reliable
   \item Doesn't scale well to massive ($>$1000) core counts
  \item Model must be made available in LALSimulation (need low level C to interface)
  \end{itemize}
  \medskip
  \pause
ILE (with RIFT extensions) (Pankow et al.(2015) and Lange et al.(2018))
\begin{itemize}
\item Fast highly parallelized algorithm  
\item Easily extended with other models (using Python)
\end{itemize}
\medskip
\pause
  gwin (in development, Capano et al.) 
  \begin{itemize}
  \item Next gen python package for parameter estimation
  \end{itemize}

\end{frame}

\begin{frame}{Inference algorithms}
  ILE+RIFT
  \begin{itemize}
  \item Specify and separate intrinsic and extrinsic parameters
  \item Lay out grid on intrinsic parameters
     \item Independent, Parallelized, Log-Likelihood calculation over each grid
       point using Monte Carlo Integration and marginalization of extrinsic parameters
     \item Gaussian Process Regression to interpolate over grid points with high
       likelihood
      \item Setup for next iteration over new interpolated grid
      \end{itemize}
\end{frame}

\begin{frame}{MGHPCC}
  Massachusetts Green High Performance Computing Center
  \begin{itemize}
    \item Developed as a joint venture of Boston University, Harvard, MIT,
      Northeastern, and the University of Massachusetts system
%    \item Contains hundreds of thousands of cores
    \item Can be accessed by UMass students under participating faculty
    \end{itemize}
    \begin{figure}
    \begin{center}
      \includegraphics[width=.40\linewidth]{mghpcc.jpg}
      \caption{The MGHPCC facility at Holyoke, Massachusetts}
    \end{center}
  \end{figure}
\end{frame}

\begin{frame}{Progress so far}
  \begin{itemize}
  \item Interfacing the new model with the codebase
  \item Performing test runs on a local cluster at RIT
  \item Setting up the codebase at MGHPCC to begin PE runs
  \end{itemize}
  \end{frame}

  \begin{frame}{Preliminary Results}
    
    \begin{itemize}
    \item Preliminary is the key word here
      \item Signal injected and PE study done using same model
      \item Sparse grid, only 5 points per dimension
        \item SNR ~ 20

      \item $m_1$ = 56
    \item $m_2$ = 14
    \item $\chi_{1z}$ = -0.5
    \item $\chi_{2z}$ = 0.5
    \end{itemize}
    
  \end{frame}

  \begin{frame}
    \begin{figure}
      \begin{center}
        \includegraphics[width=.70\linewidth]{mspin.png}
        \caption{}
      \end{center}
    \end{figure}
    \end{frame}

  \begin{frame}
    \begin{figure}
      \begin{center}
        \includegraphics[width=.70\linewidth]{mcetachiE.png}
        \caption{}
      \end{center}
    \end{figure}
    \end{frame}

    \begin{frame}{Conclusion}
      Remarks
      \begin{itemize}
      \item First PE study with this new model
       \item NR surrogates provide fast and accurate waveform models
    \item ILE+RIFT performs rapid parameter estimation using parallelized
      computation in order of hours
\end{itemize}
   
      \medskip
      Future Outlook
      \begin{itemize}
      \item Investigate effects of higher harmonic modes
      \item Understand bias in PE studies using statistical techniques
        \item Compare with other models
        \end{itemize}
      \end{frame}

      
%\frametitle{Surrogate models (without matter)}
%
%\vspace{-60pt}
%\begin{small}
%\begin{columns}
%  \begin{column}{.43\linewidth}
%    \begin{figure}
%    \begin{center}
%      \includegraphics[width=.99\linewidth]{SurrogateMap_v3}
%    \end{center}
%    \end{figure}
%  \end{column}
%  \begin{column}{.57\linewidth}
%%\structure{Model building}
%\begin{itemize}
%%\item Building model-specific basis, closed-form waveform models
%\item Closed-form waveform models
%\begin{itemize}
%\item Cannon et al. (2010, 2012, 2013), Field et al. (2011, 2012), Kaye (2012), Smith et al. (2013, 2016), Doctor et al. (2017)
%\end{itemize}
%\item Nonspinning, multimode effective one body (EOB) 
%\begin{itemize}
%\item Field et al., (PRX 2014)
%\end{itemize}
%\item Spinning, single-mode EOB
%\begin{itemize}
%\item Purrer, (CQG 2014, PRD 2016)
%\end{itemize}
%\item Multi-mode numerical relativity (\red{restrictive intervals})
%\begin{itemize}
%\item Blackman, SF,+ 2015 (non-spinning),\\
% Blackman, SF,+ 2017 (5d subspace), \\
%Blackman, SF,+ 2017 (full 7d)
%\end{itemize}
%\end{itemize}
%  \end{column}
%\end{columns}
%\end{small}
%
%\vspace{-40pt}
%\red{Current work}: PEOB, extensions in time (Vijay Varma; Mon), tidal effects (Kevin Barkett; Tues)
%%\red{Describe common choices, reason for differences, how intervals effect speedup.}
%
%\end{frame}
%
%\begin{comment}
%\begin{frame}
%\frametitle{Surrogate utility}
%
%\begin{itemize}
%\item EOB surrogate models enable speed up factors of between $10^2$ - $10^3$
%\item NR surrogate models as fast as closed-form models
%\item Over the past two years, their usage has gone from none to extensive
%\begin{itemize}
%\item EOB surrogate model of Purrer used in official LIGO searches/PE. 
%\item NR models used in studies of GW memory, parameter estimation, with others underway
%\item With more GW observations at higher SNR, NR surrogates should become increasingly important 
%\end{itemize}
%\end{itemize}
%
%\vspace{-20pt}
%
%\begin{figure}
%\begin{center}
%  \includegraphics[width=.45\linewidth]{Speedup_v2SS_mass_etas}
%  \includegraphics[width=.42\linewidth]{EOB_ExampleTiming}
%\end{center}
%\end{figure}
%
%\end{frame}
%\end{comment}
%
%%\section{Building} 
%
%\begin{frame}{Top-level NR surrogate building workflow}
%
%\begin{figure}
%\begin{center}
%  \includegraphics[width=.95\linewidth]{SurrogateWorkflow}
%\end{center}
%\end{figure}
%
%\end{frame}
%
%\begin{comment}
%\begin{frame}{Parameter space of a dimensionless waveform model}
%
%\begin{small}
%\begin{columns}
%  \begin{column}{.35\linewidth}
%\begin{itemize}
%\item Mass ratio $q$
%\item Spins $\vec{\chi}_1$ and $\vec{\chi}_2$
%\item Sampling strategies 
%\begin{itemize}
%\item Greedy (reduced basis) 
%\item Model feedback
%\item PN-sampler
%\item Sparse grid 
%\item Randomized
%\end{itemize}
%\end{itemize}
%  \end{column}
%  \begin{column}{.62\linewidth}
%\begin{figure}
%\begin{center}
%  \includegraphics[width=.95\linewidth]{BBH_diagram}
%\end{center}
%\end{figure}
%  \end{column}
%\end{columns}
%
%\begin{itemize}
%\item For physical waveforms, the total mass can be scaled back
%\item All NR models of duration $4,500$M before the peak amplitude
%%\item Modes are either $\ell \leq 3$ (5d), $\ell \leq 4$ (7d), $\ell \leq 8$ (1d)
%\end{itemize}
%\end{small}
%
%\end{frame}
%
%
%\begin{frame}{Top-level NR surrogate building workflow}
%
%\begin{figure}
%\begin{center}
%  \includegraphics[width=.95\linewidth]{SurrogateWorkflow}
%\end{center}
%\end{figure}
%
%\medskip
%\begin{itemize}
%\item Even with good sampling strategies, coverage is sparse (\red{Bad!})
%\item \red{Key}: Transform/decompose the data to look boring (minimal variation)
%\end{itemize}
%
%\end{frame}
%\end{comment}
%
%\begin{frame}{Waveform decomposition+approximation (non-spinning)}
%
%\begin{columns}
%  \begin{column}{.44\linewidth}
%\begin{figure}
%\begin{center}
%  \includegraphics[width=.99\linewidth]{EOB_ExampleDEIM.pdf}
%\end{center}
%\medskip
%\medskip
%Amplitude/phase decomposition: 
%\begin{align*}
%h(t;q) = A(t;q) \exp\left(-\mathrm{i}\phi(t;q)\right)
%\end{align*}
%\end{figure}
%  \end{column}
%  \begin{column}{.55\linewidth}
%\begin{figure}
%\begin{center}
%  \includegraphics[width=.75\linewidth]{EOB_ExampleAmpPhi.pdf}
%\end{center}
%\end{figure}
%  \end{column}
%\end{columns}
%
%\end{frame}
%
%% this block is from FullTalk.tex [useful for showing EIM in context of simpler case]
%\begin{comment}
%\begin{frame}
%\frametitle{Setup}
%
%\structure{Surrogate waveform model} is defined by 
%%$h_{\rm S}(t,\theta,\phi; q) = \sum_{\ell,m} h^{\ell, m}_{\rm S}(t; q) {}_{-2}Y_{\ell m} \left(\theta, \phi \right)$ where
%\begin{align*}
%\begin{split}
%       h_{\rm S}(t,\theta,\phi; q) & = \sum_{\ell,m} h^{\ell, m}_{\rm S}(t; q) {}_{-2}Y_{\ell m} \left(\theta, \phi \right) \, \\
%	h^{\ell, m}_{\rm S}(t; q) & = A^{\ell, m}_{\rm S} (t; q) e^{-i \varphi^{\ell, m}_{\rm S} (t; q)}  \, ,   \\
%	X_{\rm S}^{\ell, m} (t; q) &= \sum_{i=1}^{N_X} \red{B^{\ell, m}_{X,i}} (t) \blue{X^{\rm \ell, m}_i} (q)  \, , ~~ X = \{A, \varphi\} .
%\end{split}
%\end{align*}
%
%\begin{itemize}
%\item \red{$\{B^{\ell m}_{X,i} \}_{i=1}^{N_X}$}: model-specific basis 
%functions for the amplitudes, $\{B^{\ell m}_{A,i} \}_{i=1}^{N_A}$, and the phases $\{B^{\ell m}_{\varphi,i} \}_{i=1}^{N_{\varphi}}$
%\item \blue{$X^{\ell m}_i(q)$}: Functions which approximate the parametric variation of the amplitudes, $A^{\ell, m}(\green{T^{\ell m}_{A,i}}; q)$,
%and phases, $\varphi^{\ell, m}(\green{T^{\ell m}_{\varphi,i}}; q)$ {\em at a fixed time}
%\item \green{$\{ T^{\ell m}_{X,i} \} _{i=1}^{N_X}$}: the {\em empirical} interpolation nodes/times (Maday, 2009) which guarantees accurate interpolation with the basis
%
%\end{itemize}
%
%\end{frame}
%
%
%\begin{frame}
%\frametitle{Idealization}
%
%Suppose we could train our surrogate with arbitrarily many numerical relativity waveforms...
%
%\medskip
%\medskip
%\structure{Existing procedure} (Phys.Rev. X4 (2014) 3, 031006)
%\begin{enumerate}
%\item Greedy algorithm to build the basis
%\item Empirical interpolation algorithm to find the interpolation times
%\item Fits (polynomial) of the amplitude and phase data
%\end{enumerate}
%
%\end{frame}
%
%\begin{frame}
%\frametitle{Surrogate model schematic}
%
%\vspace{-10pt}
%\begin{small}
%\begin{columns}
%  \begin{column}{.3\linewidth}
%    \begin{figure}
%    \begin{center}
%      \includegraphics[width=1.09\linewidth]{cartoon_4} 
%    \end{center}
%    \end{figure}
%  \end{column}
%  \begin{column}{.65\linewidth}
%    \red{1.} Amplitude/phase basis \red{$B^{\ell, m}_{A,i} (t)$, 
%$B^{\ell, m}_{\varphi,i} (t)$} built from the NR waveforms
%themselves \red{$h(t;q_i)$}\\[5pt]
%%parameters \red{$q_i$}
%%built from \red{$h_i^{\rm RB}(t)$}
%\pause
%    \blue{2.} Specially selected times \blue{$T^{\ell m}_{A,i}$, $T^{\ell m}_{\varphi,i}$}
%uniquely determined by the basis\\[5pt]
%\pause
%    \blue{3.} Parametric fits \blue{$A^{\ell m}_i(q)$, $\varphi^{\ell m}_i(q)$} at each \blue{$T^{\ell m}_{A,i}$, $T^{\ell m}_{\varphi,i}$}\\[5pt]
%\pause
%    \yellow{4.} Evaluate the surrogate at parameter value (yellow dot) by i) evaluating the fits at each \blue{$T_i$} which ii) specifies the full waveform as:
%%through an (empirical) interpolant:
%%
%%    \begin{align*}
%%     h^{\rm S}_{\mu}(t) \equiv \sum_{i=1}^m B_i (t) h^{\rm FIT}_{\mu}(T_i) \, 
%%    \end{align*}
%%where $\{B_i\}$ is built from \red{$h_i^{\rm RB}(t)$}
%\begin{align*}
%\begin{split}
%  A_{\rm S}^{\ell, m} (t; q) &= \sum_{i=1}^{N_A} B^{\ell, m}_{A,i} (t) A^{\rm \ell, m}_i(q)  \, , \\
%  \varphi_{\rm S}^{\ell, m} (t; q) &= \sum_{i=1}^{N_{\varphi}} B^{\ell, m}_{\varphi,i} (t) \varphi^{\rm \ell, m}_i(q)  \, , \\
%  h^{\ell, m}_{\rm S}(t; q) & = A^{\ell, m}_{\rm S} (t; q) e^{-i \varphi^{\ell, m}_{\rm S} (t; q)}  \, , \\
%  h_{\rm S}(t,\theta,\phi; q) & = \sum_{\ell,m} h^{\ell, m}_{\rm S}(t; q) {}_{-2}Y_{\ell m} \left(\theta, \phi \right) \, .
%\end{split}
%\end{align*}
%  \end{column}
%\end{columns}
%\end{small}
%
%\end{frame}
%\end{comment}
%
%
%\begin{comment}
%\begin{figure}
%\begin{center}
%  \includegraphics[width=.70\linewidth]{SurrogateWaveformDecomposed.png}
%\end{center}
%\end{figure}
%
%\medskip
%\begin{itemize}
%\item NR waveform data is decomposed into easier-to-model pieces
%\item High-dimensional (least squares) fit for each piece [\red{approximation}]
%\item Reassemble pieces to evaluate the model
%\end{itemize}
%\end{comment}
%
%\begin{frame}{Waveform decomposition+approximation (precessing)}
%
%\begin{columns}
%  \begin{column}{.7\linewidth}
%    \begin{figure}
%    \begin{center}
%      \includegraphics[width=.95\linewidth]{SurrogateWaveformDecomposed.png}
%    \end{center}
%    \end{figure}
%  \end{column}
%  \begin{column}{.3\linewidth}
%\begin{itemize}
%\item NR waveform data is decomposed into easier-to-model pieces
%\item High-dimensional (least squares) fit for each piece [\red{approximation}]
%\item Reassemble pieces to evaluate the model
%\item Cross-validation study to assess model error
%\end{itemize}
%  \end{column}
%\end{columns}
%
%\end{frame}
%
%\begin{comment}
%\begin{frame}{Top-level NR surrogate building workflow}
%
%\begin{figure}
%\begin{center}
%  \includegraphics[width=.95\linewidth]{SurrogateWorkflow}
%\end{center}
%\end{figure}
%
%\medskip
%\begin{itemize}
%\item How accurate is the surrogate?
%\end{itemize}
%
%\end{frame}
%
%\begin{frame}
%\frametitle{Cross-validation (mismatch) studies}
%
%\medskip
%\medskip
%For set of 886 training NR waveforms we...
%
%\pause
%\begin{enumerate}
%\item Randomly divide into 20 sets of $44$ or $45$  \pause
%\item We build a {\em trial} surrogate using 19 of these 20 sets, the held-out set is for validation \pause
%\item Evaluate the trial surrogate and compare to the validation set (repeating this 19 more times) \pause
%\item These cross-validation errors are an honest prediction of the error
%\end{enumerate}
%
%\begin{figure}
%\begin{center}
%  \includegraphics[width=.65\linewidth]{mismatch_hist.pdf}
%\end{center}
%\end{figure}
%
%\end{frame}
%
%%\section{Public NR models}
%
%\begin{frame}{Where are the public models?}
%
%\red{Data}\\ 
%\url{https://www.black-holes.org/data/surrogates}\\[15pt]
%
%\medskip
%\medskip
%\red{Evaluation code}\\
%GWSurrogate (\url{https://pypi.python.org/pypi/gwsurrogate/})\\
%\url{https://www.black-holes.org/data/surrogates/}
%
%\end{frame}
%
%\begin{frame}{What is ``surrogate data"?}
%
%To use a surrogate model you need a (i) surrogate data file and (ii) the code (ie. rules) for evaluation. 
%
%\medskip
%\medskip
%\medskip
%\medskip
%\structure{Data}
%\begin{itemize}
%\item Training values (SpEC parameter values)
%\item Basis for temporal dependence (reduced basis or singular value basis)
%\item Coefficients for decomposed data piece's representation in this basis 
%\item Coefficients for the parameter fits for each decomposed data piece
%\end{itemize}
%
%\end{frame}
%
%
%\begin{frame}{What are the public models?}
%
%\begin{columns}
%  \begin{column}{.39\linewidth}
%{\small
%\structure{1d non-spinning}
%\begin{itemize}
%\setlength{\itemsep}{1pt}
%\item $q \in [1,10]$
%\item $\ell \leq 8$ modes
%\item Used $22$ SpEC simulations
%\end{itemize}
%
%\smallskip
%\structure{5d precessing} (GW150914-like)
%\begin{itemize}
%\setlength{\itemsep}{1pt}
%\item $q \in [1,2]$, $| \chi_i | \leq  0.8$
%\item $\vec{\chi}_2 \propto \hat{L}$  (initially)
%\item $\ell \leq 3$ modes
%\item Used $276$ SpEC simulations
%\end{itemize}
%
%\smallskip
%\structure{7d precessing}
%\begin{itemize}
%\setlength{\itemsep}{1pt}
%\item $q \in [1,2]$, $| \chi_i | \leq  0.8$
%\item $\ell \leq 4$ modes
%\item Used $744$ SpEC simulations
%\end{itemize}
%}
%  \end{column}
%  \begin{column}{.58\linewidth}
%    \begin{figure}
%    \begin{center}
%%      \includegraphics[width=1\linewidth]{waveform}
%%\caption{The largest surrogate error (1d, nonspinning) is for $q \approx 2$, for which the $(2,2)$ mode is shown.}
%\includegraphics[width=1\linewidth]{GenericPrecessingWaveform.png}
%\caption{Generic precessing waveform from 7d NR surrogate model evaluation}
%    \end{center}
%    \end{figure}
%  \end{column}
%  \end{columns}
%\end{frame}
%
%\end{comment}
%
%\begin{frame}{GWSurrogate Python package}
%
%\red{Goals}:
%\begin{itemize}
%\item Surrogate-building codes and data are model-specific (sometimes very different)
%\item GWSurrogate: easy to install, easy to use, Python\{2,3\}-based
%\item Current catalog of surrogate models + data access tools
%\item Low-level API allows access of modes, basis functions, fits, and other surrogate data
%\end{itemize}
%
%\medskip
%Why not just use LALSimulation? \pause Some models will be ported (code review underway!), but...
%\begin{itemize}
%\item Not everyone can or should need to install LALSimulation to use surrogates
%\item Its unlikely that for each new surrogate there will be LALSimulation counterpart
%\item Having multiple codes to evaluate the same model is good for the community
%\item GWSurrogate allows for low-level access of surrogate data
%\end{itemize}
%
%\end{frame}
%
%
%\begin{comment}
%\begin{ipython}[]
%>>> import gwsurrogate as gws
%>>> gws.catalog.list() # view all available models, example...
%NRSur4d2s_TDROM_grid12...
%  url: https://zenodo.org/record/1215824/files/NRSur4d2s_TDROM_grid12.h5
%  Description: Fast time-domain surrogate model for binary black hole mergers where the
%               black holes may be spinning, but the spins are restricted to a parameter
%               subspace which includes some but not all precessing configurations.
%  References: https://journals.aps.org/prd/abstract/10.1103/PhysRevD.95.104023
%>>> gws.catalog.pull('NRSur4d2s_TDROM_grid12') # download the surrogate data
%>>> spec = gws.EvaluateSurrogate('NRSur4d2s_TDROM_grid12.h5') # create surrogate model spec
%>>> modes_list, time, hp, hc = spec([1.50, 0.60, .505, .2, 0.7])
%\end{ipython}
%\end{comment}
%
%%\begin{verbatim}
%%>>> pip install gwsurrogate
%%\end{verbatim}
%
%\begin{frame}[fragile]{Quick start (installation + catalog)}
%
%{\footnotesize
%Installation: 
%\begin{verbatim}
%>>> pip install gwsurrogate 
%\end{verbatim}
%
%\medskip
%\pause
%Query the catalog: 
%\begin{ipython}[]
%>>> import gwsurrogate as gws
%>>> gws.catalog.list()
%NRSur4d2s_TDROM_grid12...
%  url: https://zenodo.org/record/1215824/files/NRSur4d2s_TDROM_grid12.h5
%  Description: Fast (spline) time-domain surrogate model. Spins are restricted to a 
%               subspace which includes some but not all precessing configurations.
%  References: https://journals.aps.org/prd/abstract/10.1103/PhysRevD.95.104023
%>>> gws.catalog.pull("NRSur4d2s_TDROM_grid12") # Download the surrogate data
%\end{ipython}
%}
%
%Currently, 6 surrogate models are available. Each has a name, dataset url, description, and reference. 
%
%\end{frame}
%
%\begin{comment}
%\begin{frame}{Quickstart}
%
%\begin{itemize}
%\item \url{https://pypi.python.org/pypi/gwsurrogate}
%\begin{itemize}
%\item $>>>$ pip install gwsurrogate
%\medskip
%\medskip
%\end{itemize}
%\pause
%\item Data files can be downloaded from a public location
%\begin{itemize}
%\item Non-spinning multi-mode effective one body surrogates
%\item Multi-modal non-spinning numerical relativity SpEC surrogate
%\item Recent 5d and 7d models are being implemented (\red{Coming soon!})
%\end{itemize}
%\item Low-level API allows direct access of modes, basis functions, and fits
%\begin{itemize}
%\item Leveraged to accelerate the ILE parameter estimation pipeline (O'Shaughnessy, SF+ 2017)
%\end{itemize}
%\end{itemize}
%
%\end{frame}
%\end{comment}
%
%
%\begin{frame}[fragile]{Quick start (evaluation)}
%
%{\footnotesize
%\begin{ipython}[]
%>>> import gwsurrogate as gws
%>>> spec_precessing     = gws.EvaluateSurrogate("NRSur4d2s_TDROM_grid12.h5") # Load 
%>>> modes, time, hp, hc = spec_precessing([1.50, 0.60, 0.505, 0.2, 0.7]) # Evaluate
%>>> spec_nospin         = gws.EvaluateSurrogate("SpEC_q1_10_NoSpin.h5",excluded=[(8,7)]) # Load
%>>> modes, time, hp, hc = spec_nospin(q=3.14) # Evaluate
%\end{ipython}
%}
%
%\vspace{-10pt}
%\begin{figure}
%\begin{center}
%%  \includegraphics[width=.5\linewidth]{SpEC_q3pt14_22}
%  \includegraphics[width=.42\linewidth]{SurrMemMode.png}
%  \includegraphics[width=.42\linewidth]{SpEC_q3pt14_32}
%\end{center}
%\end{figure}
%
%\end{frame}
%
%\begin{frame}[fragile]{Some features}
%
%{\footnotesize
%\begin{ipython}[]
%>>> import gwsurrogate as gws
%>>> spec = gws.EvaluateSurrogate("SpEC_q1_10_NoSpin.h5",ell=[2],m=[0]) # (2,0) only
%>>> spec = gws.EvaluateSurrogate("SpEC_q1_10_NoSpin.h5",ell=[2,3],m=[2,2]) # (2,2), (3,2)
%>>> spec22 = spec.single_mode_dict["(2,2)"] # access surrogate data for (2,2)-mode
%>>> spec22.basis() # access basis functions defining the temporal dependence
%>>> spec22.eim_coeffs() # Parametric dependence of the model (fit functions) 
%
%\end{ipython}
%}
%
%\medskip
%\medskip
%Directly access surrogate data...
%\begin{itemize}
%\item Basis functions defining the temporal dependence
%\item Parametric dependence of the model in closed-form (analytic derivatives)
%\item ... and any additional data defining the model; mode-by-mode
%\end{itemize}
%
%\end{frame}
%
%\begin{frame}
%
%\structure{Remarks}
%\begin{itemize}
%\item Surrogate and reduced order modeling offers an exciting new approach to overcome a variety of computationally challenging problems of GW physics
%%\item Evaluations are fast and accurate
%\item Publicly available surrogate evaluation package GWSurrogate 
%\begin{itemize}
%\item Numerous Jupyter tutorial notebooks
%\item Active development (version 0.7 released last night)
%\item Long-term plans: host on github -- happy to have feedback (open an issue)
%%\item \url{https://pypi.python.org/pypi/gwsurrogate/}
%\end{itemize}
%\end{itemize}
%
%\medskip
%\structure{Future outlook}
%\begin{itemize}
%\item Extending the range of validity of NR surrogates
%\item As new models are built they will be included into the surrogate catalog
%\item Contributions are welcomed! If you've built a surrogate model, we can happily add it to the catalog
%%\item Precessing EOB surrogates
%%\item NR surrogates will become increasingly important to GW data analysis efforts
%%\item NR surrogates are being implemented in LIGO 
%%\item Many open questions. Lots of work to be done!
%\end{itemize}
%
%\end{frame}


\end{document}

